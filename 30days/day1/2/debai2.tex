% !TeX program = xelatex
\documentclass[11pt, a4paper]{article}

% --- CẤU HÌNH NGÔN NGỮ & FONT ---
\usepackage[vietnamese, provide=*]{babel}
\usepackage{fontspec}

% Cấu hình font chuẩn cho máy Windows
\babelfont{rm}{Times New Roman} 
\babelfont{sf}{Arial}           

% --- CÁC GÓI CẦN THIẾT ---
\usepackage[a4paper, top=2cm, bottom=2cm, left=2cm, right=2cm]{geometry}
\usepackage{amsmath}
\usepackage{booktabs}
\usepackage{tabularx}
\usepackage{multicol}
\usepackage{fancyhdr}
\usepackage{lastpage}
\usepackage{array}
\usepackage{enumitem}

% --- CẤU HÌNH TRANG & ĐỊNH DẠNG ---
\setlist[itemize]{label=-, noitemsep, topsep=0pt} % Gạch đầu dòng, gọn dòng
\setlength{\parindent}{0pt} % Không thụt đầu dòng
\setlength{\parskip}{6pt}   % Khoảng cách giữa các đoạn

\pagestyle{fancy}
\fancyhf{}
\fancyfoot[C]{Trang \thepage/\pageref{LastPage}}
\renewcommand{\headrulewidth}{0pt}

% Định nghĩa cột C cho tabularx để canh giữa
\newcolumntype{Y}{>{\centering\arraybackslash}X}

\begin{document}

\begin{center}
    \textbf{\Large ĐỀ THI LẬP TRÌNH THI ĐẤU} \\
    \textit{Thời gian làm bài: 180 phút}
\end{center}
\hrule
\vspace{0.5cm}

% ----------------------------------------------------------------------
\section*{Bài 1: Số Đẹp (5 điểm)}

\textbf{Đề bài:}
Một số nguyên dương được gọi là \textbf{số đẹp} nếu như nó chia hết cho 5 và tổng các chữ số của nó cũng chia hết cho 5.

\textbf{Yêu cầu:} Cho dãy $n$ số nguyên dương $a_{1},a_{2},...,a_{n}$. Hãy đếm xem có bao nhiêu số đẹp trong dãy trên.

\textbf{Dữ liệu (Input):}
\begin{itemize}
    \item Dòng đầu chứa số nguyên dương $n$.
    \item Tiếp theo là $n$ dòng, dòng thứ $i$ ($i=1,2,...,n$) chứa số nguyên dương $a_{i}$.
    \item Tổng số lượng các chữ số của $a_{1},a_{2},...,a_{n}$ không vượt quá $10^{6}$.
\end{itemize}

\textbf{Kết quả (Output):} In ra màn hình một số nguyên duy nhất là số lượng số đẹp trong dãy số đã cho.

\textbf{Ví dụ:}
\begin{center}
    \begin{tabularx}{0.9\textwidth}{|p{5cm}|X|X|}
        \hline
        \textbf{Dữ liệu} & \textbf{Kết quả} & \textbf{Giải thích} \\
        \hline
        5 \newline 15 \newline 50 \newline 140 \newline 25 \newline 10 & 
        2 & 
        Chỉ có 2 số \textbf{50}, \textbf{140} thoả mãn đồng thời hai điều kiện: chia hết cho 5 và tổng các chữ số cũng chia hết cho 5. \\
        \hline
    \end{tabularx}
\end{center}

\textbf{Ràng buộc:}
\begin{itemize}
    \item Có 80\% số tests ứng với 80\% số điểm của bài thoả mãn $a_{i} \le 10^9$ $\forall i=1,2,...,n$.
    \item Các tests còn lại không có ràng buộc bổ sung.
\end{itemize}

\vspace{0.5cm}
\hrule
\vspace{0.5cm}

% ----------------------------------------------------------------------
\section*{Bài 2: Tổng đoạn con liên tiếp lớn nhất (5 điểm)}

\textbf{Đề bài:}
Cho dãy $n$ số nguyên $a_{1},a_{2},...,a_{n}$. Một dãy con liên tiếp của dãy trên là một bộ các phần tử đứng kề nhau trong dãy, có dạng $a_{i},a_{i+1},...,a_{j}$ với $1 \le i \le j \le n$. Hãy tìm tổng lớn nhất trong tất cả các dãy con liên tiếp của dãy đã cho.

\textbf{Dữ liệu (Input):}
\begin{itemize}
    \item Dòng đầu chứa số nguyên dương $n$ $(n \le 10^{6})$.
    \item Dòng thứ hai chứa $n$ số nguyên lần lượt là $a_{1},a_{2},...,a_{n}$ ($|a_i| \le 10^9$).
\end{itemize}

\textbf{Kết quả (Output):} Ghi ra màn hình một số nguyên duy nhất là giá trị tổng lớn nhất tìm được.

\textbf{Ví dụ:}
\begin{center}
    \begin{tabularx}{0.6\textwidth}{|X|Y|}
        \hline
        \textbf{Dữ liệu} & \textbf{Kết quả} \\
        \hline
        5 \newline 2 -1 5 -3 4 & 7 \\
        \hline
    \end{tabularx}
\end{center}

\textbf{Ràng buộc:}
\begin{itemize}
    \item 40\% số tests: $n \le 500$.
    \item 20\% số tests tiếp theo: $n \le 5000$.
    \item Các tests còn lại: $n \le 10^{6}$.
\end{itemize}

\newpage

% ----------------------------------------------------------------------
\section*{Bài 3: Đếm cặp (6 điểm)}

\textbf{Đề bài:}
Cho dãy $n$ số nguyên $a_{1},a_{2},...,a_{n}$ và số nguyên dương $M$. Hãy đếm số lượng cặp $(i,j)$ với $1 \le i < j \le n$ sao cho $a_{i}+a_{j}$ chia hết cho $M$.

\textbf{Dữ liệu (Input):}
\begin{itemize}
    \item Dòng đầu chứa hai số nguyên dương $n, M$ $(n \le 3 \times 10^{5}, M \le 10^{18})$.
    \item Dòng thứ hai chứa $n$ số nguyên $a_{1},a_{2},...,a_{n}$ $(|a_{i}| \le 10^{18})$.
\end{itemize}

\textbf{Kết quả (Output):} Ghi ra màn hình một số nguyên duy nhất là số cặp tìm được.

\textbf{Ví dụ:}
\begin{center}
    \begin{tabularx}{0.9\textwidth}{|p{5cm}|X|X|}
        \hline
        \textbf{Dữ liệu} & \textbf{Kết quả} & \textbf{Giải thích} \\
        \hline
        5 4 \newline 1 3 2 6 2 & 4 & Các cặp $(i,j)$ là: (1,2), (3,4), (3,5), (4,5). \\
        \hline
    \end{tabularx}
\end{center}

\textbf{Ràng buộc:}
\begin{itemize}
    \item 40\% số tests: $n \le 5000$.
    \item 20\% số tests tiếp theo: $M \le 10^{6}$.
    \item Các tests còn lại không có ràng buộc bổ sung.
\end{itemize}

\vspace{0.5cm}
\hrule
\vspace{0.5cm}

% ----------------------------------------------------------------------
\section*{Bài 4: Chọn hoa (6 điểm)}

\textbf{Đề bài:}
Cho $n$ bông hoa có độ đẹp lần lượt là $a_{1},a_{2},...,a_{n}$ và một số nguyên $K$. Bạn cần chọn ra hai tập hợp bông hoa (gọi là hai bó hoa) sao cho mỗi bông hoa chỉ thuộc tối đa một bó.

Điều kiện của mỗi bó hoa là chênh lệch giữa độ đẹp của bông hoa lớn nhất và nhỏ nhất trong bó đó không được vượt quá $K$. Hãy tìm tổng số lượng bông hoa lớn nhất có thể chọn được cho cả hai bó hoa.

\textbf{Dữ liệu (Input):}
\begin{itemize}
    \item Dòng đầu chứa hai số nguyên dương $n$ và $K$ $(n \le 10^{6}, K \le 10^{9})$.
    \item Dòng thứ hai chứa $n$ số nguyên $a_{1},a_{2},...,a_{n}$ $(1 \le a_{i} \le 10^{9})$.
\end{itemize}

\textbf{Kết quả (Output):} Ghi ra màn hình một số nguyên duy nhất là tổng số lượng bông hoa lớn nhất tìm được.

\textbf{Ví dụ:}
\begin{center}
    \begin{tabularx}{0.9\textwidth}{|p{5cm}|X|X|}
        \hline
        \textbf{Dữ liệu} & \textbf{Kết quả} & \textbf{Giải thích} \\
        \hline
        6 3 \newline 1 10 2 7 12 4 & 5 & Bó 1: $\{1,2,4\}$ (chênh lệch $4-1=3 \le 3$). \newline Bó 2: $\{10, 12\}$ hoặc $\{7, 10\}$. \newline Tổng là $3+2=5$. \\
        \hline
    \end{tabularx}
\end{center}

\textbf{Ràng buộc:}
\begin{itemize}
    \item Subtask 1 (40\% số điểm): $n \le 10$.
    \item Subtask 2 (20\% số điểm): $n \le 100$.
    \item Subtask 3 (20\% số điểm): $n \le 5000$.
    \item Subtask 4 (20\% số điểm): $n \le 10^{6}$.
\end{itemize}

\newpage

% ----------------------------------------------------------------------
\section*{Bài 5: Tưới cây (8 điểm)}

\textbf{Đề bài:}
Trường THCS nơi Dũng đang học có trồng một hàng cây xanh gồm $n$ cây được đánh số thứ tự từ 1 đến $n$ (từ trái sang phải). Có thể coi hàng cây như trục tọa độ $Ox$ và cây thứ $i$ có tọa độ $x_{i}$ ($x_{1}<x_{2}<...<x_{n}$).

Để tưới nước, nhà trường lắp đặt $m$ vòi tưới tự động. Vòi thứ $i$ $(i=1..m)$ được lắp tại vị trí cây $t_{i}$, có bán kính tưới là $R_{i}$. Vòi này tưới được cây $t_{i}$ và tất cả các cây có khoảng cách đến $t_{i}$ không vượt quá $R_{i}$.

\textbf{Yêu cầu:} Cho biết vị trí lắp đặt $m$ vòi nước và bán kính tưới nước. Hãy đếm xem có bao nhiêu cây được tưới nước.

\textbf{Dữ liệu (Input):}
\begin{itemize}
    \item Dòng đầu tiên chứa hai số nguyên dương $n, m$ $(1 \le m \le n \le 10^{6})$.
    \item Dòng thứ hai chứa $n$ số nguyên $x_{1},x_{2},...,x_{n}$ ($0 < x_i \le 10^9$) lần lượt là tọa độ của các cây.
    \item Tiếp theo là $m$ dòng, dòng thứ $i$ chứa hai số nguyên dương $t_{i}, R_{i}$ ($1 \le t_{i} \le n; R_{i} \le 10^{9}$) lần lượt là số hiệu cây đặt vòi và bán kính tưới.
\end{itemize}

\textbf{Kết quả (Output):} Ghi ra màn hình một số nguyên duy nhất là số cây được tưới nước.

\textbf{Ví dụ:}
\begin{center}
    \begin{tabularx}{0.9\textwidth}{|p{5cm}|X|X|}
        \hline
        \textbf{Dữ liệu} & \textbf{Kết quả} & \textbf{Giải thích} \\
        \hline
        5 2 \newline 1 3 5 7 9 \newline 1 1 \newline 4 2 & 5 & Vòi 1 (tại cây 1, tọa độ 1, BK 1) tưới được cây: 1 (tọa độ 1), 2 (tọa độ 3). \newline Vòi 2 (tại cây 4, tọa độ 7, BK 2) tưới được cây: 3 (tọa độ 5), 4 (tọa độ 7), 5 (tọa độ 9). \newline Tổng: Cả 5 cây đều được tưới. \\
        \hline
    \end{tabularx}
\end{center}

\textbf{Ràng buộc:}
\begin{itemize}
    \item 30\% số tests: $m=1$.
    \item 30\% số tests tiếp theo: $n \le 2000$.
    \item Các tests còn lại không có giới hạn bổ sung.
\end{itemize}

\end{document}