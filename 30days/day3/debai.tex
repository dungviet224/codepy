\documentclass[11pt, a4paper]{article}

% --- UNIVERSAL PREAMBLE BLOCK ---
\usepackage[a4paper, top=2cm, bottom=2cm, left=2cm, right=2cm]{geometry}
\usepackage{fontspec}

\usepackage[vietnamese, provide=*]{babel}

\babelprovide[import]{vietnamese}
\babelprovide[import]{english}

% Set default/Latin font to Sans Serif in the main (rm) slot
\babelfont{rm}{Noto Sans}

% Add because main language is not English
\usepackage{enumitem}
\setlist[itemize]{label=-}

% --- PACKAGES ---
\usepackage{amsmath}
\usepackage{booktabs}
\usepackage{tabularx}
\usepackage{multicol}
\usepackage{fancyhdr}
\usepackage{lastpage}
\usepackage{array}

% --- PAGE SETUP ---
\pagestyle{fancy}
\fancyhf{}
\fancyfoot[C]{Trang \thepage/\pageref{LastPage}}
\renewcommand{\headrulewidth}{0pt}

% --- CUSTOM COMMANDS ---
\newcommand{\bai}[2]{\vspace{0.5cm}\noindent\textbf{Bài #1: #2}}
\newcommand{\iobox}[2]{
    \begin{center}
    \begin{tabularx}{0.8\textwidth}{|X|X|}
    \hline
    \textbf{Input} & \textbf{Output} \\
    \hline
    \vtop{#1} & \vtop{#2} \\
    \hline
    \end{tabularx}
    \end{center}
}

\begin{document}

% --- HEADER ---
\begin{center}
    \begin{tabular}{>{\centering\arraybackslash}p{0.4\textwidth} >{\centering\arraybackslash}p{0.5\textwidth}}
        SỞ GIÁO DỤC VÀ ĐÀO TẠO & \textbf{KỲ THI CHỌN HỌC SINH GIỎI CẤP TỈNH} \\
        LẠNG SƠN & \textbf{LỚP 11 THPT NĂM HỌC 2025 - 2026} \\
        \textbf{ĐỀ THI CHÍNH THỨC} & Môn thi: \textbf{TIN HỌC (ĐẠI TRÀ)} \\
        (Đề thi có 02 trang) & Ngày thi: \textbf{01/02/2026} \\
         & Thời gian làm bài: \textbf{180 phút} (không kể thời gian giao đề)
    \end{tabular}
\end{center}
\hrule
\vspace{0.5cm}

\textbf{Tổng quan đề thi (Ngày 3):}
\begin{center}
\begin{tabular}{|l|l|l|l|}
\hline
\textbf{Câu} & \textbf{Tên bài} & \textbf{File bài làm} & \textbf{Điểm} \\ \hline
1 & Số Phong Phú & RICHNUM.CPP / .PY & 5.0 \\ \hline
2 & Nén Xâu Ký Tự & STRZIP.CPP / .PY & 6.0 \\ \hline
3 & Vùng Ảnh Liên Thông & IMAGE.CPP / .PY & 6.0 \\ \hline
4 & Chọn Quà Lưu Niệm & GIFT.CPP / .PY & 3.0 \\ \hline
\end{tabular}
\end{center}

\vspace{0.5cm}

% --- BÀI 1 ---
\bai{1}{Số Phong Phú (5.0 điểm)}

Trong số học, một số nguyên dương $X$ được gọi là \textbf{Số phong phú} (Abundant Number) nếu tổng các ước số thực sự của nó (các ước số nhỏ hơn chính nó) lớn hơn nó.
Ví dụ: Số 12 có các ước thực sự là 1, 2, 3, 4, 6. Tổng là $1+2+3+4+6 = 16$. Vì $16 > 12$ nên 12 là số phong phú.

\textbf{Yêu cầu:} Cho dãy số nguyên dương $A$ gồm $N$ phần tử. Hãy đếm xem trong dãy có bao nhiêu số là Số phong phú.

\textbf{Dữ liệu vào (File: RICHNUM.INP):}
\begin{itemize}
    \item Dòng đầu tiên chứa số nguyên $N$ ($1 \le N \le 10^5$).
    \item Dòng thứ hai chứa $N$ số nguyên $A_1, A_2, \dots, A_N$ ($1 \le A_i \le 10^5$).
\end{itemize}

\textbf{Kết quả (File: RICHNUM.OUT):}
\begin{itemize}
    \item Ghi ra số lượng Số phong phú tìm được trong dãy.
\end{itemize}

\textbf{Ví dụ:}
\iobox{
5 \newline
12 10 18 20 7
}{
3
}
\textit{Giải thích: Các số phong phú là 12, 18, 20.
- 10: 1+2+5=8 < 10 (Không phải).
- 18: 1+2+3+6+9=21 > 18 (Phải).
- 7: 1 < 7 (Không phải).}

% --- BÀI 2 ---
\bai{2}{Nén Xâu Ký Tự (6.0 điểm)}

Để tiết kiệm dung lượng lưu trữ, người ta sử dụng phương pháp nén xâu RLE (Run-Length Encoding) cho các xâu ký tự chỉ chứa chữ cái in hoa. Quy tắc nén như sau: Nếu có $K$ ký tự giống nhau liên tiếp ($K \ge 1$), ta sẽ thay thế dãy đó bằng số nguyên $K$ theo sau là ký tự đó.
Ví dụ: Xâu \texttt{AAABBCCCCA} sẽ được nén thành \texttt{3A2B4C1A}.

\textbf{Yêu cầu:} Cho một xâu ký tự $S$. Hãy in ra xâu nén của nó theo quy tắc trên.

\textbf{Dữ liệu vào (File: STRZIP.INP):}
\begin{itemize}
    \item Một dòng duy nhất chứa xâu $S$ gồm các chữ cái in hoa (độ dài không quá $10^5$).
\end{itemize}

\textbf{Kết quả (File: STRZIP.OUT):}
\begin{itemize}
    \item Ghi ra xâu sau khi đã nén.
\end{itemize}

\textbf{Ví dụ:}
\iobox{
AAABBC
}{
3A2B1C
}
\textit{Giải thích: 3 chữ A liên tiếp $\to$ 3A, 2 chữ B liên tiếp $\to$ 2B, 1 chữ C $\to$ 1C.}

\newpage

% --- BÀI 3 ---
\bai{3}{Vùng Ảnh Liên Thông (6.0 điểm)}

Một bức ảnh đen trắng được biểu diễn bởi ma trận kích thước $M \times N$ gồm các số 0 (màu trắng) và 1 (màu đen). Một "vùng ảnh" là tập hợp các ô màu đen (số 1) liên thông với nhau (kề nhau theo cạnh trái, phải, trên, dưới).

\textbf{Yêu cầu:} Hãy đếm số lượng vùng ảnh (số lượng vùng liên thông các số 1) có trong ma trận và xác định diện tích (số ô) của vùng ảnh lớn nhất.

\textbf{Dữ liệu vào (File: IMAGE.INP):}
\begin{itemize}
    \item Dòng đầu tiên chứa hai số nguyên $M, N$ ($1 \le M, N \le 1000$).
    \item $M$ dòng tiếp theo, mỗi dòng chứa $N$ số nguyên (0 hoặc 1).
\end{itemize}

\textbf{Kết quả (File: IMAGE.OUT):}
\begin{itemize}
    \item Dòng 1: Ghi số lượng vùng ảnh.
    \item Dòng 2: Ghi diện tích của vùng ảnh lớn nhất (nếu không có số 1 nào thì ghi 0).
\end{itemize}

\textbf{Ví dụ:}
\iobox{
5 6 \newline
1 1 0 0 0 1 \newline
1 0 0 1 1 1 \newline
0 0 0 1 0 0 \newline
1 1 0 1 0 1 \newline
1 1 0 0 0 1
}{
4 \newline
6
}
\textit{Giải thích:
- Vùng 1: Góc trên trái (3 ô).
- Vùng 2: Góc trên phải và giữa (6 ô) - Đây là vùng lớn nhất.
- Vùng 3: Góc dưới trái (4 ô).
- Vùng 4: Góc dưới phải (2 ô đơn lẻ nối nhau? Không, ô (4,6) và (5,6) là 2 ô dọc. Ô (1,6)-(2,6) là vùng 2. Vùng 4 là (4,6)-(5,6)).
Tổng cộng 4 vùng.}

% --- BÀI 4 ---
\bai{4}{Chọn Quà Lưu Niệm (3.0 điểm)}

Trong chuyến tham quan Lạng Sơn, An ghé vào một cửa hàng lưu niệm. Cửa hàng có $N$ món đồ, món thứ $i$ có giá tiền là $W_i$ và độ yêu thích là $V_i$. An mang theo số tiền tối đa là $T$. An muốn chọn mua một số món đồ sao cho tổng giá tiền không vượt quá $T$ và tổng độ yêu thích là lớn nhất có thể. Mỗi món đồ chỉ được mua tối đa 1 lần.

\textbf{Yêu cầu:} Hãy tính tổng độ yêu thích lớn nhất mà An có thể đạt được.

\textbf{Dữ liệu vào (File: GIFT.INP):}
\begin{itemize}
    \item Dòng đầu chứa hai số nguyên $N$ và $T$ ($1 \le N \le 20$, $1 \le T \le 10^9$).
    \item $N$ dòng tiếp theo, mỗi dòng chứa hai số nguyên $W_i, V_i$ ($1 \le W_i, V_i \le 10^6$).
\end{itemize}

\textbf{Kết quả (File: GIFT.OUT):}
\begin{itemize}
    \item Ghi ra tổng độ yêu thích lớn nhất tìm được.
\end{itemize}

\textbf{Ví dụ:}
\iobox{
4 10 \newline
4 4 \newline
3 3 \newline
5 6 \newline
2 1
}{
10
}
\textit{Giải thích: Chọn món thứ 1 (giá 4, thích 4) và món thứ 3 (giá 5, thích 6). Tổng tiền: $4+5=9 \le 10$. Tổng độ thích: $4+6=10$.
(Lưu ý: Với $N \le 20$, học sinh có thể sử dụng phương pháp Quay lui/Vét cạn. Nếu $N$ lớn hơn cần dùng Quy hoạch động, nhưng ở mức độ đại trà $N$ thường nhỏ để kiểm tra kỹ năng đệ quy/duyệt nhị phân).}

\vspace{1cm}
\begin{center}
    \textbf{---------------- HẾT ----------------} \\
    \textit{Thí sinh không được sử dụng tài liệu. Cán bộ coi thi không giải thích gì thêm.}
\end{center}

\end{document}